\documentclass{beamer}
\usepackage[T1]{fontenc}
\usepackage[utf8]{inputenc}
\usepackage{lmodern}
\usepackage[francais]{babel}
\usepackage{graphicx}
\usepackage{beamerthemeWarsaw}
\expandafter\def\expandafter\insertshorttitle\expandafter{\insertshorttitle\hfill\insertframenumber\,/\,\inserttotalframenumber}

\title{Algorithmes distribués sur Android}
\author{Kevin \textsc{Boulala}, Maxime \textsc{Dubois}}
\institute{Université de Franche Comté}
\date{4 février 2016}

\begin{document}

  \begin{frame}
    \titlepage
  \end{frame}

  \begin{frame}
    \setcounter{tocdepth}{1}
    \tableofcontents[]
  \end{frame}
  
  \section{Présentation générale}
  \begin{frame}
    \frametitle{Présentation générale}
    \begin{block}{L'objectif principal}
      Développement d'une plateforme permettant de programmer simplement des algorithmes distribués et de les exécuter sur des appareils android.
    \end{block}
    \begin{block}{Les autres objectifs}
      \begin{itemize}
        \item Utiliser cette plateforme pour implanter des algorithmes distribués
        \item Développer une application utilisant ces algorithmes distribués
      \end{itemize}
    \end{block}
  \end{frame}
  
  \section{Choix technologiques}
  \begin{frame}
    \setcounter{tocdepth}{2}
    \tableofcontents[currentsection]
  \end{frame}
    \subsection{NFC}
    \begin{frame}
      \frametitle{Choix technologiques}
      \framesubtitle{NFC}
      \begin{block}{Near Field Communication}
        \begin{itemize}
          \item Communication en Champ Proche (CCP) en français
          \item Éxplosion de son utilisation en 2012
          \item Android compatible depuis la version Ice Cream Sandwich (4.0)
          \item Connexion des appareils très facile
          \item 4cm de portée
          \item 424kbit/s de débit
          \item Pas utilisé pour le transfert de fichiers, plutôt pour établir les connexions
        \end{itemize}
      \end{block}
    \end{frame}
    \subsection{Bluetooth}
    \begin{frame}
      \frametitle{Choix technologiques}
      \framesubtitle{Bluetooth}
      \begin{block}{Bluetooth 4 (2010)}
        \begin{itemize}
          \item Quelques chiffres : création en 1994, > 28000 membres
          \item Dans la continuité de Bluetooth 3 (2009) :
          \begin{itemize}
            \item $ \approx $ 10 mètres de portée (classe 2)
            \item 24Mbit/s de débit
          \end{itemize}
          \item Utilisé pour les communications, le transfert de fichiers
        \end{itemize}
      \end{block}
    \end{frame}
    \subsection{Wi-Fi Direct}
    \begin{frame}
      \frametitle{Choix technologiques}
      \framesubtitle{Wi-Fi Direct}
      \begin{block}{Wi-Fi Direct anciennement Wi-Fi P2P}
        \begin{itemize}
          \item Wi-Fi mais sans point d'accès intermédiaire
          \item 60 mètres de portée
          \item 250Mbit/s de débit
          \item Pour la même utilisation que le Bluetooth pour les appareils Android
        \end{itemize}
      \end{block}
    \end{frame}
    \subsection{Bilan}
    \begin{frame}
      \frametitle{Choix technologiques}
      \framesubtitle{Bilan}
      \begin{small}
        \begin{block}{Synthèse}
          \begin{itemize}
            \item {\small Élimination du NFC : portée trop réduire}
          \end{itemize}
          \begin{table}
            \begin{center}
              \scalebox{0.7}{
              \begin{tabular}{|c|c|c|}
                \hline
                & \textbf{Bluetooth} & \textbf{Wi-Fi Direct} \\
                \hline
                \textbf{Vitesse (en Mbit/s)} & 24 & 250 \\
                \hline
                \textbf{Portée (en m)} & 10 & 60 \\
                \hline
              \end{tabular}
              }
            \end{center}
          \end{table}
        \end{block}
        \begin{block}{Solution retenue}
          \begin{itemize}
            \item Wi-Fi Direct nécessite que l'utilisateur valide chaque connexion
            \item Quelques instabilités avec le Wi-Fi Direct et nos appareils
          \end{itemize}
          \begin{center}
            $ \Rightarrow $ Pour ces raisons, nous avons retenu le \textbf{Bluetooth}.
          \end{center}
        \end{block}
      \end{small}
    \end{frame}
  
  \section{Construction du réseaux}
  \begin{frame}
    \setcounter{tocdepth}{2}
    \tableofcontents[currentsection]
  \end{frame}
    \subsection{Les objectifs}
    \begin{frame}
      \frametitle{Construction du réseaux}
      \framesubtitle{Les objectifs}
      \begin{small}
        \begin{block}{Les objectifs}
          \begin{itemize}
            \item Connexions automatiques
            \item Ignorer les appareils qui ne nous intéressent pas
          \end{itemize}
          $ \Rightarrow $ Pas de topologie classique des réseaux Bluetooth (maître - esclaves)
          \begin{columns}
            \begin{column}{.5\textwidth}
              \begin{figure}
                \begin{center}
                  \includegraphics[width=.4\textwidth]{images/BluetoothPiconet.png}
                  \label{fig:piconet}
                  \caption{Un piconet}
                \end{center}
              \end{figure}
            \end{column}
            \begin{column}{.5\textwidth}
              \begin{figure}
                \begin{center}
                  \includegraphics[width=.4\textwidth]{images/BluetoothScatternet.png}
                  \label{fig:scatternet}
                  \caption{Un scatternet}
                \end{center}
              \end{figure}
            \end{column}
          \end{columns}
        \end{block}
      \end{small}
    \end{frame}
    \begin{frame}
      \frametitle{Construction du réseaux}
      \framesubtitle{Les objectifs}
      \begin{block}{Pas de maître, ni d'esclave}
        \begin{figure}
          \begin{center}
            \includegraphics[width=.4\textwidth]{images/reseau_quelconque.png}
            \caption{Un réseau quelconque}
            \label{fig:quelconque}
          \end{center}
        \end{figure}
      \end{block}
    \end{frame}
    \subsection{Connexion automatique}
    \begin{frame}
      \frametitle{Connexion automatique}
      \framesubtitle{Identification et adressage}
      \begin{block}{Adresse MAC}
        \begin{itemize}
          \item Bluetooth est un protocole réseau n'utilisant pas IP
          \item Bluetooth utilise des adresses MAC
          \item Pour identifier les appareils, nous utilisons nous aussi les adresses MAC
        \end{itemize}
      \end{block}
    \end{frame}
    \begin{frame}
      \frametitle{Connexion automatique}
      \framesubtitle{Les socket bluetooth dans Android}
      \begin{block}{Les comportements}
        \begin{itemize}
          \item Les données transférées restent intactes ou les erreurs sont détectées
          \item Les données transférées arrivent dans le même ordre dans lequel elles ont été envoyés
        \end{itemize}
      \end{block}
      \begin{block}{Mode de connexion : secure ou insecure}
        \begin{itemize}
          \item « insecure » ne demande pas de confirmation à l'utilisateur contrairement à « secure »
          \item Une fois la connexion établit, les échanges sont chiffrés dans les deux modes
        \end{itemize}
      \end{block}
    \end{frame}
    \begin{frame}
      \frametitle{Connexion automatique}
      \framesubtitle{Les socket bluetooth dans Android}
      \begin{block}{Les problèmes}
        \begin{itemize}
          \item Ne peut faire du multi-client éfficacement et simplement \\
          $ \Rightarrow $ Un thread de réception, un thread d'envoi
          \item Les exceptions sont toutes du type IOException \\
          $ \Rightarrow $ Il faut alors analyser le message de l'exception
        \end{itemize}
      \end{block}
    \end{frame}
    \begin{frame}
      \frametitle{Connexion automatique}
      \framesubtitle{Rendre l'appareil visible}
      \begin{block}{La solution classique}
        L'API Android propose une solution pour rendre un appareil découvrable.
        \begin{itemize}
          \item Nécessite la validation manuelle de l'utilisateur
          \item Ne dure qu'un certain temps
        \end{itemize}
      \end{block}
      \begin{block}{Notre solution}
        Nous utilisons l'introspection de Java pour accéder à une API cachée et appeler la méthode setScanMode qui supprime les limitations de la solution classique. Mais...
        \begin{itemize}
          \item aucune assurance que cette méthode reste viable dans les prochaines versions d'Android
        \end{itemize}
      \end{block}
    \end{frame}
    \begin{frame}
      \frametitle{Connexion automatique}
      \framesubtitle{Détecter des appareils}
      \begin{block}{Détection}
        L'API Android fournit une méthode startDiscovery() qui lance une recherche pendant 12 secondes. Nous remarquons cependant que cette découverte est coûteuse :
        \begin{itemize}
          \item consomme de l'énergie
          \item sollicite beaucoup le bluetooth rendant le tout instable (peut provoquer des déconnexions)
        \end{itemize}
        Il faut donc utiliser cette fonctionnalité judicieusement.
      \end{block}
    \end{frame}
    \begin{frame}
      \frametitle{Connexion automatique}
      \framesubtitle{Comment se connecter aux bons appareils ?}
      \begin{block}{Récupération des UUIDs d'un appareil}
        \begin{itemize}
          \item UUID : Universal Unique Identifier
          \item Permet l'identification des services
          \item Mais
          \begin{itemize}
            \item la liste des UUIDs récupérés étaient souvent incomplètes ou vides
            \item le mécanisme coûteux en bande passante
          \end{itemize}
        \end{itemize}
        \begin{center}
          $ \Rightarrow $ On a donc abandonné cette solution
        \end{center}
      \end{block}
    \end{frame}
    \begin{frame}
      \frametitle{Connexion automatique}
      \framesubtitle{Comment se connecter aux bons appareils ?}
      \begin{block}{Solution de remplacement de la récupération des UUIDs}
        \begin{itemize}
          \item On ne cherche plus à savoir si un périphérique possède une UUID précise
          \item L'application se connecte directement et on attend un timeout
        \end{itemize}
      \end{block}
    \end{frame}
    \subsection{Solution proposée}
    \begin{frame}
      \frametitle{La solution proposée}
      \framesubtitle{}
      \begin{block}{Stabilisation du réseau}
        \begin{itemize}
          \item Lancer une seule découverte par périphérique lors du démarrage de l'application
          \item Limité à 4 connexions pour s'assurer une certaine stabilité
        \end{itemize}
      \end{block}
      \begin{block}{Problème}
        \begin{itemize}
          \item Lorsqu'un périphérique se déconnecte, le réseau peut s'éffondrer aléatoirement
        \end{itemize}
      \end{block}
    \end{frame}
    \begin{frame}
      \frametitle{La solution proposée}
      \framesubtitle{}
      \begin{block}{Réseau en forme d'arbre}
        \begin{figure}
          \begin{center}
            \includegraphics[width=.5\textwidth]{images/auto.png}
            \caption{Exemple de réseau}
            \label{fig:auto}
          \end{center}
        \end{figure}
      \end{block}
    \end{frame}

  \section{Routage}
  \begin{frame}
    \frametitle{Routage}
    \framesubtitle{}
  
  \end{frame}
  
  \section{API}
  \begin{frame}
    \frametitle{API}
    \framesubtitle{}
  
  \end{frame}
  
  % TODO La suite dépend de ce qu'on aura produit d'ici jeudi.
  
\end{document}
